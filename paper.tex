\documentclass[12pt]{report}
\usepackage[
  dissertation
 ,final
 ,raggedbottom
%,tocbold        % uncomment to enable bold chapter titles in the ToC
                 %
                 % the style guidelines state that page numbers in the
                 % ToC should not be bold, but leave it up to the author
                 % and specific department guidelines as to how the
                 % chapter (or other section) titles should be typeset.
]{USCthesis}

% guidelines for manuscript formatting: https://graduateschool.usc.edu/wp-content/uploads/2020/11/Manuscript_Formatting_and_Documentation_Styles.pdf

%% our customizations %%%%%%%%%%%%%%%%%%%%%%%%%%%%%%%%%%%%%%%%%%%%%%%%%%
\usepackage[export]{adjustbox} % for frame option in \includegraphics
\usepackage{amsmath}
\usepackage{amssymb}
\usepackage{array}
\usepackage[utf8]{inputenc} % load inputenc before csquotes
\usepackage[english]{babel}
\usepackage[
  backend     = biber,
  citestyle   = authoryear,
  doi         = true,
  hyperref    = true,
  maxbibnames = 99,
  natbib      = true,
  sortlocale  = en_US,
  style       = authoryear,
]{biblatex}
\usepackage{booktabs}
\usepackage{color, colortbl}
\usepackage{csquotes}
\usepackage{efbox}
\usepackage{enumitem}
\usepackage[shortcuts]{extdash} % use `\-/' to hyphenate words/phrases that have a dash in them
% \usepackage[tt=false]{libertine} % libertine's \ttfamily isn't that great
\usepackage[T1]{fontenc} % load fonts before fontenc
%\usepackage[symbol]{footmisc}
\usepackage{footmisc}
\usepackage[
  showframe = false,% draw a border around textwidth
  pass      = true, % force 8.5"x11" pagesize
]{geometry}
\usepackage{graphicx}
%\usepackage[notquote]{hanging} % enables negative indents in paragraphs
\usepackage{hyphenat}
\usepackage{ifthen}
\usepackage{lipsum}
\usepackage{multirow}
\usepackage{parnotes}
\usepackage{pdflscape} % rotate some pages in an {landscape} environment
\usepackage{pifont}
\usepackage{ragged2e}
\usepackage{seqsplit}
\usepackage{siunitx}
\usepackage{subcaption}
\usepackage{tabularx}
\usepackage{xcolor}
\usepackage{xspace}
\usepackage{url}

\usepackage[
  bookmarks     = true,
  breaklinks    = true,
  colorlinks    = true,
  hypertexnames = false,
  pdfpagelabels = false,
  citecolor     = {blue!80!black},
  linkcolor     = {blue!80!black},
  urlcolor      = {blue!80!black},
]{hyperref} % load hyperref as the last package

% pkg: biblatex
\setlength\bibitemsep{0.5\baselineskip}                 % add a line between entries
\AtEveryBibitem{\iffieldundef{doi}{}{\clearfield{url}}} % if DOI, hide URL

% \addbibresource{paper.bib}

% pkg: siunitx
% some guidelines https://physics.nist.gov/cuu/Units/checklist.html
\sisetup{
  tight-spacing  = true
  ,detect-family = true
  ,detect-mode   = true
  ,binary-units  = true    % support for MB, GB, etc.
  ,range-units   = single  % "3% to 5%" -> "3 to 5%"
  ,range-phrase  = --      % "3 to 5%"  -> "3--5%"
}

% pkg: babel, hyperref
\addto\extrasenglish{%
  \renewcommand{\chapterautorefname}{Chapter}
  \renewcommand{\sectionautorefname}{Section}
  \renewcommand{\subsectionautorefname}{Section}
  \renewcommand{\subsubsectionautorefname}{Section}
}

% pkg: url
\renewcommand{\UrlFont}{\footnotesize\tt}

% our custom commands
\renewcommand{\ttdefault}{cmtt} % use computer modern for teletype

%%% draft mode / toggle commands %%%
\usepackage{etoolbox}
\newtoggle{draft}
\settoggle{draft}{true} % change toggle for draft or final versions

\iftoggle{draft}{
  % if 'draft' toggle is true
  \overfullrule=10pt                       % highlight overfull hboxes
}{
  % if 'draft' toggle is false
  \PassOptionsToPackage{final}{showlabels} % hide labels on figures, etc
}

% if you're including existing papers into your thesis, it helps to put
% content behind a toggle (or conditional) so you only have to maintain
% and keep consistency on one copy. see "introduction.tex".
\newtoggle{thesis}
\settoggle{thesis}{true}

\usepackage[inline]{showlabels}
\renewcommand{\showlabelfont}{\sffamily \color{blue}}
\renewcommand{\showlabelsetlabel}[1]{\efbox{\showlabelfont #1}}
%%%%%%%%%%%%%%%%%%%%%%%%%%%%%%%%%%%%%%%%%%%%%%%%%%%%%%%%%%%%%%%%%%%%%%%%

%% sboysel customization %%%%%%%%%%%%%%%%%%%%%%%%%%%%%%%%%%%%%%%%%%%%%%%

%\usepackage[margin=1in]{geometry}
%\usepackage{lipsum}
%\usepackage{abstract}
\usepackage{amsfonts}
%\usepackage{amssymb}
\usepackage{amsthm}
%\usepackage{amsmath}
\usepackage{bm}
\usepackage{bbm}
\usepackage{breqn}
\usepackage{epigraph}
\usepackage{tikz}
\usepackage{tikz-network}
\usepackage{pgfplots}
\pgfplotsset{compat=1.18}
\usepackage{xcolor}
\usepackage{setspace}
\usepackage{enumitem}
\usepackage[mathscr]{euscript}
\usepackage{subcaption}
\usepackage{graphicx}
\usepackage{subfiles}
%\usepackage[round]{natbib}
%\usepackage{hyperref}
%\hypersetup{
%    colorlinks=true,
%    linkcolor=blue,
%    citecolor=blue,
%    anchorcolor=blue,
%    filecolor=magenta,
%    urlcolor=cyan,
%    pdftitle={Digital Supply Chains [Boysel 2022]}
%}
%\setlength{\parindent}{0pt}
%\setlength{\parskip}{1em}
%\usepackage{titlesec}
%\titlespacing\section{0pt}{12pt plus 4pt minus 2pt}{0pt plus 2pt minus 2pt}
%\titlespacing\subsection{0pt}{12pt plus 4pt minus 2pt}{0pt plus 2pt minus 2pt}
%\titlespacing\subsubsection{0pt}{12pt plus 4pt minus 2pt}{0pt plus 2pt minus 2pt}
%\usepackage{fancyhdr}
\usepackage{caption}
%%\captionsetup[table]{skip=10pt}
\usepackage{tabularx}
\usepackage{threeparttable}
\usepackage{array}
\newcolumntype{Y}{>{\centering\arraybackslash}X}
\newcolumntype{d}[1]{D{.}{.}{#1}} 
\usepackage{booktabs}
%\usepackage{lastpage}

%%% Theorem Environments
\newtheorem{theorem}{Theorem}[section]
%\newtheorem{acknowledgement}{Acknowledgement}
\newtheorem{assumption}{Assumption}
%\newtheorem{algorithm}[theorem]{Algorithm}
%\newtheorem{axiom}[theorem]{Axiom}
%\newtheorem{case}[theorem]{Case}
%\newtheorem{claim}[theorem]{Claim}
%\newtheorem{conclusion}[theorem]{Conclusion}
%%\newtheorem{condition}[theorem]{Condition}
%\newtheorem{conjecture}[theorem]{Conjecture}
%\newtheorem{corollary}[theorem]{Corollary}
%\newtheorem{criterion}[theorem]{Criterion}
%\newtheorem{definition}[theorem]{Definition}
\newtheorem{example}[theorem]{Example}
%\newtheorem{exercise}[theorem]{Exercise}
%\newtheorem{lemma}[theorem]{Lemma}
%\newtheorem{notation}[theorem]{Notation}
%\newtheorem{problem}[theorem]{Problem}
\newtheorem{proposition}{Proposition}
\newtheorem*{remark}{Remark}
%\newtheorem{solution}[theorem]{Solution}
%\newtheorem{summary}[theorem]{Summary}

%%% New Macros
\renewcommand\qed{$\blacksquare$}
\newcommand{\st}{\, s.t. \,}
\newcommand{\grad}{\bigtriangledown}
\DeclareMathOperator*{\argmax}{arg\,max}
\DeclareMathOperator*{\argmin}{arg\,min}
\DeclareMathSymbol{\shortminus}{\mathbin}{AMSa}{"39}
\newcommand{\Cov}{\mathrm{Cov}}

%% URLs %%%
\urldef\urllibiosourcerank\url{https://docs.libraries.io/overview.html#sourcerank}

\newif\ifdissertation
\dissertationtrue     % (or \dissertationfalse for the standalone)

%% chapter appendices %%%
%%%% source: https://tex.stackexchange.com/q/120716/188684

\usepackage[titles]{tocloft}
%\setcounter{tocdepth}{2}
\usepackage{appendix}
\usepackage{chngcntr}
\usepackage{etoolbox}

% Start of subappendices environment
\AtBeginEnvironment{subappendices}{%
\chapter*{Appendices}
\addcontentsline{toc}{section}{Appendices}
\counterwithin{figure}{section}
\counterwithin{table}{section}
}
% End of subappendices environment
\AtEndEnvironment{subappendices}{%
\counterwithout{figure}{section}
\counterwithout{table}{section}
}

%% Bibliography %%%
\addbibresource{../intro/references.bib}
\addbibresource{../pub-goods/references.bib}
\addbibresource{../formation/references.bib}

%% Figures %%%
\graphicspath{{../pub-goods/figures/}{../formation/figures/}}

%%%%%%%%%%%%%%%%%%%%%%%%%%%%%%%%%%%%%%%%%%%%%%%%%%%%%%%%%%%%%%%%%%%%%%%%

%%% front matter %%%%%%%%%%%%%%%%%%%%%%%%%%%%%%%%%%%%%%%%%%%%%%%%%%%%%%%
\begin{document}

% title should be all caps
\title{SUSTAINING OPEN SOURCE SOFTWARE PRODUCTION: AN EMPIRICAL ANALYSIS THROUGH
THE LENS OF MICROECONOMICS}

% use your full name!
% https://cs.stanford.edu/~knuth/news19.html
% "Let's celebrate everybody's full names"
\author{Samuel Jospeh Boysel}

% major should be all caps
\majorfield{ECONOMICS}

% date should be May, August, or December (when degrees are conferred)
\submitdate{December 2022}

%%% preface %%%%%%%%%%%%%%%%%%%%%%%%%%%%%%%%%%%%%%%%%%%%%%%%%%%%%%%%%%%%
\begin{preface}
  \prefacesection{Dedication}
  To my mother, father, Claire, and Jake, who give my work meaning.


  \prefacesection{Acknowledgements}
  First and foremost, I am deeply indebted to my advisor Matthew Kahn, whose
support and guidance throughout my academic career has been second to none. I am
also grateful for the detailed feedback on this work provided by my dissertation
committee members David Kempke, Paulina Oliva, and Robert Metcalfe.  I
appreciate the work my committee has done in providing high-level insight,
catching errors or problems, and suggesting strategies to achieve my research
objectives. Additionally, I thank Shane Greenstein for his thorough review of
early iterations of my work. Moreover, conversations with Professors Cheng
Hsiao, Jeff Weaver, Vittorio Bassi, Monica Morlacco, and Geert Ridder helped
refine my thought process and methodology. I'd like to also thank the unyielding
support of Young Miller and Annie Le, whose tireless efforts within the USC
economics department too often go without proper acknowledgement.

Finally, comments from colleagues Rajat Kochar, Ruozi Song, Nicolas Roig, Karim
Fajury, Thomas Ash, Islamul Haque, Amy Mahler, Yue Fang, Liying Yang, Xiongfei
Li, Taraq Khan, and Amanda Ang have all helped shaped my research in positive
directions. More importantly, I cherish their friendship in ways that transcend
research.



  {
  \hypersetup{hidelinks} % color all links black in the preface
  \tableofcontents
  % fix numwidth for subappendix figs/tables
  \makeatletter
    \def\l@table{\@dottedtocline{1}{1em}{4em}}
    \def\l@figure{\@dottedtocline{1}{1em}{4em}}
    \onehalfspacing
  \makeatother
  \listoftables
  \listoffigures
  }

  \prefacesection{Abstract}
  %%\lipsum[2-3]

In this manuscript, we explore microeconomic behavior shaping the production of
open source software (OSS).  We fortify our analysis with economic structure to
guide our narrative and assess our hypotheses empirically, filling important
gaps in the literature on the supply side of markets for OSS goods.  Our
motivation is rooted in a desire to better understand how various microeconomic
phenomena influence sustained development of widely used OSS infrastructure.
Following an introduction in Chapter~\ref{ch:introduction}, our contribution is
divided into two distinct chapters.

In Chapter~\ref{ch:pg}, we examine the extent to which peer effects influence
the private provision of public goods.  In the case of public information goods,
peer contribution may facilitate or otherwise incentivize further contribution
from others, effectively subsidizing private provision.  We first utilize a
reduced form approach to derive causal estimates of net peer effects in public
goods contribution by exploiting a peers-of-peers identification strategy.  We
next develop a structural model of peer-influenced public good provision that
both (1) separates extensive and intensive margin contribution decisions and (2)
decomposes contribution into marginal private benefits and costs.  We apply
these methodologies using a sample of peer contribution histories for 2,287 OSS
projects hosted on the GitHub collaboration platform.  Both reduced form and
structural approaches suggest peer effects are much stronger along the extensive
margin than the intensive margin.  Contemporaneous intensive margin effects,
while heterogenous across time and projects, are small and centered around zero,
suggesting that strategic complementarity and substitution in peer contribution
likely offset one another.  Our counterfactual analysis suggests (extensive
margin) peer effects account for nearly 56\% of cumulative aggregate
contribution for our sample, which translates to a value-added of 1-1.5 million
software developer labor hours.  These results support the notion that OSS is
largely developed by disproportionate efforts from smaller groups of dedicated
core maintainers, who integrate incremental contributions from the wider
community, and casts doubt on the potential for peer effects alone to deliver
sustained maintenance labor to individual projects.

In Chapter~\ref{ch:dsc}, we turn our attention to the formation of software
dependency networks.  Developers of software projects can leverage the
functionality of existing open source projects.  This practice can potentially
lower the cost of development albeit at the inherent risk of relying on external
components. A ``downstream'' project maintainer can choose to ``import''
elements of an ``upstream'' project to outsource functionality, but is uncertain
how future changes in this dependency project may expose her own project to
software faults or vulnerabilities. Software dependency networks therefore
represent a ``digital supply chain'', an ecosystem of interdependent public
goods that confer an intricate set of both positive and negative externalities
for project maintainers and end users.  Focusing on microeconomic fundamentals
of the dependency management problem faced by the risk averse project
maintainer, we use both reduced form and structural approaches to study how
dependency networks create value, what forces shape their formation, and how
individual behavior can influence the robustness of equilibrium network
structure.  We use a sample of open source software projects from the Node.js
JavaScript packaging ecosystem for which contribution and dependency formation
decisions are observed in real-time. Finally, we consider several policy
interventions that can improve equilibrium welfare.  In particular, we find that
removing less that 1\% of core projects can reduce aggregate project quality by
more than 5\% for the remaining peers.


\end{preface}

%%% introduction %%%%%%%%%%%%%%%%%%%%%%%%%%%%%%%%%%%%%%%%%%%%%%%%%%%%%%%
\chapter{Introduction}\label{ch:introduction}

%\graphicspath{}
\input{../intro/intro.tex}

%%% chapter %%%%%%%%%%%%%%%%%%%%%%%%%%%%%%%%%%%%%%%%%%%%%%%%%%%%%%%%%%%%

% if you're "stapling" together papers, it's easy to include your paper
% directory by way of symlinks, or copying the entire paper as a
% subdirectory.
%
% for example, if your paper directory looks like the following:
%
%   foobar/          - top level paper directory
%   foobar/fig/      - where all graphics and figures live
%   foobar/paper.bib - bibliography
%   foobar/paper.tex - monolithic .tex file for paper
%
% then you might use the folloiwng:
%
%   \graphicspath{foobar/fig}
%   \addbibresource{foobar/paper.bib}
%   \documentclass[11pt]{report}
\usepackage[
  dissertation
 ,final
 ,raggedbottom
%,tocbold        % uncomment to enable bold chapter titles in the ToC
                 %
                 % the style guidelines state that page numbers in the
                 % ToC should not be bold, but leave it up to the author
                 % and specific department guidelines as to how the
                 % chapter (or other section) titles should be typeset.
]{USCthesis}

% guidelines for manuscript formatting: https://graduateschool.usc.edu/wp-content/uploads/2020/11/Manuscript_Formatting_and_Documentation_Styles.pdf

%% our customizations %%%%%%%%%%%%%%%%%%%%%%%%%%%%%%%%%%%%%%%%%%%%%%%%%%
\usepackage[export]{adjustbox} % for frame option in \includegraphics
\usepackage{amsmath}
\usepackage{amssymb}
\usepackage{array}
\usepackage[utf8]{inputenc} % load inputenc before csquotes
\usepackage[english]{babel}
\usepackage[
  backend     = biber,
  citestyle   = authoryear,
  doi         = true,
  hyperref    = true,
  maxbibnames = 99,
  natbib      = true,
  sortlocale  = en_US,
  style       = authoryear,
]{biblatex}
\usepackage{booktabs}
\usepackage{color, colortbl}
\usepackage{csquotes}
\usepackage{efbox}
\usepackage{enumitem}
\usepackage[shortcuts]{extdash} % use `\-/' to hyphenate words/phrases that have a dash in them
% \usepackage[tt=false]{libertine} % libertine's \ttfamily isn't that great
\usepackage[T1]{fontenc} % load fonts before fontenc
%\usepackage[symbol]{footmisc}
\usepackage{footmisc}
\usepackage[
  showframe = false,% draw a border around textwidth
  pass      = true, % force 8.5"x11" pagesize
]{geometry}
\usepackage{graphicx}
%\usepackage[notquote]{hanging} % enables negative indents in paragraphs
\usepackage{hyphenat}
\usepackage{ifthen}
\usepackage{lipsum}
\usepackage{multirow}
\usepackage{parnotes}
\usepackage{pdflscape} % rotate some pages in an {landscape} environment
\usepackage{pifont}
\usepackage{ragged2e}
\usepackage{seqsplit}
\usepackage{siunitx}
\usepackage{subcaption}
\usepackage{tabularx}
\usepackage{xcolor}
\usepackage{xspace}
\usepackage{url}

\usepackage[
  bookmarks     = true,
  breaklinks    = true,
  colorlinks    = true,
  hypertexnames = false,
  pdfpagelabels = false,
  citecolor     = {blue!80!black},
  linkcolor     = {blue!80!black},
  urlcolor      = {blue!80!black},
]{hyperref} % load hyperref as the last package

% pkg: biblatex
\setlength\bibitemsep{0.5\baselineskip}                 % add a line between entries
\AtEveryBibitem{\iffieldundef{doi}{}{\clearfield{url}}} % if DOI, hide URL

% \addbibresource{paper.bib}

% pkg: siunitx
% some guidelines https://physics.nist.gov/cuu/Units/checklist.html
\sisetup{
  tight-spacing  = true
  ,detect-family = true
  ,detect-mode   = true
  ,binary-units  = true    % support for MB, GB, etc.
  ,range-units   = single  % "3% to 5%" -> "3 to 5%"
  ,range-phrase  = --      % "3 to 5%"  -> "3--5%"
}

% pkg: babel, hyperref
\addto\extrasenglish{%
  \renewcommand{\chapterautorefname}{Chapter}
  \renewcommand{\sectionautorefname}{Section}
  \renewcommand{\subsectionautorefname}{Section}
  \renewcommand{\subsubsectionautorefname}{Section}
}

% pkg: url
\renewcommand{\UrlFont}{\footnotesize\tt}

% our custom commands
\renewcommand{\ttdefault}{cmtt} % use computer modern for teletype

%%% draft mode / toggle commands %%%
\usepackage{etoolbox}
\newtoggle{draft}
\settoggle{draft}{true} % change toggle for draft or final versions

\iftoggle{draft}{
  % if 'draft' toggle is true
  \overfullrule=10pt                       % highlight overfull hboxes
}{
  % if 'draft' toggle is false
  \PassOptionsToPackage{final}{showlabels} % hide labels on figures, etc
}

% if you're including existing papers into your thesis, it helps to put
% content behind a toggle (or conditional) so you only have to maintain
% and keep consistency on one copy. see "introduction.tex".
\newtoggle{thesis}
\settoggle{thesis}{true}

\usepackage[inline]{showlabels}
\renewcommand{\showlabelfont}{\sffamily \color{blue}}
\renewcommand{\showlabelsetlabel}[1]{\efbox{\showlabelfont #1}}
%%%%%%%%%%%%%%%%%%%%%%%%%%%%%%%%%%%%%%%%%%%%%%%%%%%%%%%%%%%%%%%%%%%%%%%%

%% sboysel customization %%%%%%%%%%%%%%%%%%%%%%%%%%%%%%%%%%%%%%%%%%%%%%%

%\usepackage[margin=1in]{geometry}
%\usepackage{lipsum}
%\usepackage{abstract}
\usepackage{amsfonts}
%\usepackage{amssymb}
\usepackage{amsthm}
%\usepackage{amsmath}
\usepackage{bm}
\usepackage{bbm}
\usepackage{breqn}
\usepackage{epigraph}
\usepackage{tikz}
\usepackage{tikz-network}
\usepackage{pgfplots}
\pgfplotsset{compat=1.18}
\usepackage{xcolor}
\usepackage{setspace}
\usepackage{enumitem}
\usepackage[mathscr]{euscript}
\usepackage{subcaption}
\usepackage{graphicx}
\usepackage{subfiles}
%\usepackage[round]{natbib}
%\usepackage{hyperref}
%\hypersetup{
%    colorlinks=true,
%    linkcolor=blue,
%    citecolor=blue,
%    anchorcolor=blue,
%    filecolor=magenta,
%    urlcolor=cyan,
%    pdftitle={Digital Supply Chains [Boysel 2022]}
%}
%\setlength{\parindent}{0pt}
%\setlength{\parskip}{1em}
%\usepackage{titlesec}
%\titlespacing\section{0pt}{12pt plus 4pt minus 2pt}{0pt plus 2pt minus 2pt}
%\titlespacing\subsection{0pt}{12pt plus 4pt minus 2pt}{0pt plus 2pt minus 2pt}
%\titlespacing\subsubsection{0pt}{12pt plus 4pt minus 2pt}{0pt plus 2pt minus 2pt}
%\usepackage{fancyhdr}
\usepackage{caption}
%%\captionsetup[table]{skip=10pt}
\usepackage{tabularx}
\usepackage{threeparttable}
\usepackage{array}
\newcolumntype{Y}{>{\centering\arraybackslash}X}
\newcolumntype{d}[1]{D{.}{.}{#1}} 
\usepackage{booktabs}
%\usepackage{lastpage}

%%% Theorem Environments
\newtheorem{theorem}{Theorem}[section]
%\newtheorem{acknowledgement}{Acknowledgement}
\newtheorem{assumption}{Assumption}
%\newtheorem{algorithm}[theorem]{Algorithm}
%\newtheorem{axiom}[theorem]{Axiom}
%\newtheorem{case}[theorem]{Case}
%\newtheorem{claim}[theorem]{Claim}
%\newtheorem{conclusion}[theorem]{Conclusion}
%%\newtheorem{condition}[theorem]{Condition}
%\newtheorem{conjecture}[theorem]{Conjecture}
%\newtheorem{corollary}[theorem]{Corollary}
%\newtheorem{criterion}[theorem]{Criterion}
%\newtheorem{definition}[theorem]{Definition}
\newtheorem{example}[theorem]{Example}
%\newtheorem{exercise}[theorem]{Exercise}
%\newtheorem{lemma}[theorem]{Lemma}
%\newtheorem{notation}[theorem]{Notation}
%\newtheorem{problem}[theorem]{Problem}
\newtheorem{proposition}{Proposition}
\newtheorem*{remark}{Remark}
%\newtheorem{solution}[theorem]{Solution}
%\newtheorem{summary}[theorem]{Summary}

%%% New Macros
\renewcommand\qed{$\blacksquare$}
\newcommand{\st}{\, s.t. \,}
\newcommand{\grad}{\bigtriangledown}
\DeclareMathOperator*{\argmax}{arg\,max}
\DeclareMathOperator*{\argmin}{arg\,min}
\DeclareMathSymbol{\shortminus}{\mathbin}{AMSa}{"39}
\newcommand{\Cov}{\mathrm{Cov}}

%% URLs %%%
\urldef\urllibiosourcerank\url{https://docs.libraries.io/overview.html#sourcerank}

\newif\ifdissertation
\dissertationtrue     % (or \dissertationfalse for the standalone)

%% chapter appendices %%%
%%%% source: https://tex.stackexchange.com/q/120716/188684

\usepackage[titles]{tocloft}
%\setcounter{tocdepth}{2}
\usepackage{appendix}
\usepackage{chngcntr}
\usepackage{etoolbox}

% Start of subappendices environment
\AtBeginEnvironment{subappendices}{%
\chapter*{Appendices}
\addcontentsline{toc}{section}{Appendices}
\counterwithin{figure}{section}
\counterwithin{table}{section}
}
% End of subappendices environment
\AtEndEnvironment{subappendices}{%
\counterwithout{figure}{section}
\counterwithout{table}{section}
}

%% Bibliography %%%
\addbibresource{../intro/references.bib}
\addbibresource{../pub-goods/references.bib}
\addbibresource{../formation/references.bib}

%% Figures %%%
\graphicspath{{../pub-goods/figures/}{../formation/figures/}}

%%%%%%%%%%%%%%%%%%%%%%%%%%%%%%%%%%%%%%%%%%%%%%%%%%%%%%%%%%%%%%%%%%%%%%%%

%%% front matter %%%%%%%%%%%%%%%%%%%%%%%%%%%%%%%%%%%%%%%%%%%%%%%%%%%%%%%
\begin{document}

% title should be all caps
\title{SUSTAINING OPEN SOURCE SOFTWARE PRODUCTION: AN EMPIRICAL ANALYSIS THROUGH
THE LENS OF MICROECONOMICS}

% use your full name!
% https://cs.stanford.edu/~knuth/news19.html
% "Let's celebrate everybody's full names"
\author{Samuel Jospeh Boysel}

% major should be all caps
\majorfield{ECONOMICS}

% date should be May, August, or December (when degrees are conferred)
\submitdate{December 2022}

%%% preface %%%%%%%%%%%%%%%%%%%%%%%%%%%%%%%%%%%%%%%%%%%%%%%%%%%%%%%%%%%%
\begin{preface}
  \prefacesection{Dedication}
  To my mother, father, Claire, and Jake, who give my work meaning.


  \prefacesection{Acknowledgements}
  First and foremost, I am deeply indebted to my advisor Matthew Kahn, whose
support and guidance throughout my academic career has been second to none. I am
also grateful for the detailed feedback on this work provided by my dissertation
committee members David Kempke, Paulina Oliva, and Robert Metcalfe.  I
appreciate the work my committee has done in providing high-level insight,
catching errors or problems, and suggesting strategies to achieve my research
objectives. Additionally, I thank Shane Greenstein for his thorough review of
early iterations of my work. Moreover, conversations with Professors Cheng
Hsiao, Jeff Weaver, Vittorio Bassi, Monica Morlacco, and Geert Ridder helped
refine my thought process and methodology. I'd like to also thank the unyielding
support of Young Miller and Annie Le, whose tireless efforts within the USC
economics department too often go without proper acknowledgement.

Finally, comments from colleagues Rajat Kochar, Ruozi Song, Nicolas Roig, Karim
Fajury, Thomas Ash, Islamul Haque, Amy Mahler, Yue Fang, Liying Yang, Xiongfei
Li, Taraq Khan, and Amanda Ang have all helped shaped my research in positive
directions. More importantly, I cherish their friendship in ways that transcend
research.



  {
  \hypersetup{hidelinks} % color all links black in the preface
  \tableofcontents
  % fix numwidth for subappendix figs/tables
  \makeatletter
    \def\l@table{\@dottedtocline{1}{1em}{4em}}
    \def\l@figure{\@dottedtocline{1}{1em}{4em}}
    \onehalfspacing
  \makeatother
  \listoftables
  \listoffigures
  }

  \prefacesection{Abstract}
  %%\lipsum[2-3]

In this manuscript, we explore microeconomic behavior shaping the production of
open source software (OSS).  We fortify our analysis with economic structure to
guide our narrative and assess our hypotheses empirically, filling important
gaps in the literature on the supply side of markets for OSS goods.  Our
motivation is rooted in a desire to better understand how various microeconomic
phenomena influence sustained development of widely used OSS infrastructure.
Following an introduction in Chapter~\ref{ch:introduction}, our contribution is
divided into two distinct chapters.

In Chapter~\ref{ch:pg}, we examine the extent to which peer effects influence
the private provision of public goods.  In the case of public information goods,
peer contribution may facilitate or otherwise incentivize further contribution
from others, effectively subsidizing private provision.  We first utilize a
reduced form approach to derive causal estimates of net peer effects in public
goods contribution by exploiting a peers-of-peers identification strategy.  We
next develop a structural model of peer-influenced public good provision that
both (1) separates extensive and intensive margin contribution decisions and (2)
decomposes contribution into marginal private benefits and costs.  We apply
these methodologies using a sample of peer contribution histories for 2,287 OSS
projects hosted on the GitHub collaboration platform.  Both reduced form and
structural approaches suggest peer effects are much stronger along the extensive
margin than the intensive margin.  Contemporaneous intensive margin effects,
while heterogenous across time and projects, are small and centered around zero,
suggesting that strategic complementarity and substitution in peer contribution
likely offset one another.  Our counterfactual analysis suggests (extensive
margin) peer effects account for nearly 56\% of cumulative aggregate
contribution for our sample, which translates to a value-added of 1-1.5 million
software developer labor hours.  These results support the notion that OSS is
largely developed by disproportionate efforts from smaller groups of dedicated
core maintainers, who integrate incremental contributions from the wider
community, and casts doubt on the potential for peer effects alone to deliver
sustained maintenance labor to individual projects.

In Chapter~\ref{ch:dsc}, we turn our attention to the formation of software
dependency networks.  Developers of software projects can leverage the
functionality of existing open source projects.  This practice can potentially
lower the cost of development albeit at the inherent risk of relying on external
components. A ``downstream'' project maintainer can choose to ``import''
elements of an ``upstream'' project to outsource functionality, but is uncertain
how future changes in this dependency project may expose her own project to
software faults or vulnerabilities. Software dependency networks therefore
represent a ``digital supply chain'', an ecosystem of interdependent public
goods that confer an intricate set of both positive and negative externalities
for project maintainers and end users.  Focusing on microeconomic fundamentals
of the dependency management problem faced by the risk averse project
maintainer, we use both reduced form and structural approaches to study how
dependency networks create value, what forces shape their formation, and how
individual behavior can influence the robustness of equilibrium network
structure.  We use a sample of open source software projects from the Node.js
JavaScript packaging ecosystem for which contribution and dependency formation
decisions are observed in real-time. Finally, we consider several policy
interventions that can improve equilibrium welfare.  In particular, we find that
removing less that 1\% of core projects can reduce aggregate project quality by
more than 5\% for the remaining peers.


\end{preface}

%%% introduction %%%%%%%%%%%%%%%%%%%%%%%%%%%%%%%%%%%%%%%%%%%%%%%%%%%%%%%
\chapter{Introduction}\label{ch:introduction}

%\graphicspath{}
\input{../intro/intro.tex}

%%% chapter %%%%%%%%%%%%%%%%%%%%%%%%%%%%%%%%%%%%%%%%%%%%%%%%%%%%%%%%%%%%

% if you're "stapling" together papers, it's easy to include your paper
% directory by way of symlinks, or copying the entire paper as a
% subdirectory.
%
% for example, if your paper directory looks like the following:
%
%   foobar/          - top level paper directory
%   foobar/fig/      - where all graphics and figures live
%   foobar/paper.bib - bibliography
%   foobar/paper.tex - monolithic .tex file for paper
%
% then you might use the folloiwng:
%
%   \graphicspath{foobar/fig}
%   \addbibresource{foobar/paper.bib}
%   \documentclass[11pt]{report}
\usepackage[
  dissertation
 ,final
 ,raggedbottom
%,tocbold        % uncomment to enable bold chapter titles in the ToC
                 %
                 % the style guidelines state that page numbers in the
                 % ToC should not be bold, but leave it up to the author
                 % and specific department guidelines as to how the
                 % chapter (or other section) titles should be typeset.
]{USCthesis}

% guidelines for manuscript formatting: https://graduateschool.usc.edu/wp-content/uploads/2020/11/Manuscript_Formatting_and_Documentation_Styles.pdf

%% our customizations %%%%%%%%%%%%%%%%%%%%%%%%%%%%%%%%%%%%%%%%%%%%%%%%%%
\usepackage[export]{adjustbox} % for frame option in \includegraphics
\usepackage{amsmath}
\usepackage{amssymb}
\usepackage{array}
\usepackage[utf8]{inputenc} % load inputenc before csquotes
\usepackage[english]{babel}
\usepackage[
  backend     = biber,
  citestyle   = authoryear,
  doi         = true,
  hyperref    = true,
  maxbibnames = 99,
  natbib      = true,
  sortlocale  = en_US,
  style       = authoryear,
]{biblatex}
\usepackage{booktabs}
\usepackage{color, colortbl}
\usepackage{csquotes}
\usepackage{efbox}
\usepackage{enumitem}
\usepackage[shortcuts]{extdash} % use `\-/' to hyphenate words/phrases that have a dash in them
% \usepackage[tt=false]{libertine} % libertine's \ttfamily isn't that great
\usepackage[T1]{fontenc} % load fonts before fontenc
%\usepackage[symbol]{footmisc}
\usepackage{footmisc}
\usepackage[
  showframe = false,% draw a border around textwidth
  pass      = true, % force 8.5"x11" pagesize
]{geometry}
\usepackage{graphicx}
%\usepackage[notquote]{hanging} % enables negative indents in paragraphs
\usepackage{hyphenat}
\usepackage{ifthen}
\usepackage{lipsum}
\usepackage{multirow}
\usepackage{parnotes}
\usepackage{pdflscape} % rotate some pages in an {landscape} environment
\usepackage{pifont}
\usepackage{ragged2e}
\usepackage{seqsplit}
\usepackage{siunitx}
\usepackage{subcaption}
\usepackage{tabularx}
\usepackage{xcolor}
\usepackage{xspace}
\usepackage{url}

\usepackage[
  bookmarks     = true,
  breaklinks    = true,
  colorlinks    = true,
  hypertexnames = false,
  pdfpagelabels = false,
  citecolor     = {blue!80!black},
  linkcolor     = {blue!80!black},
  urlcolor      = {blue!80!black},
]{hyperref} % load hyperref as the last package

% pkg: biblatex
\setlength\bibitemsep{0.5\baselineskip}                 % add a line between entries
\AtEveryBibitem{\iffieldundef{doi}{}{\clearfield{url}}} % if DOI, hide URL

% \addbibresource{paper.bib}

% pkg: siunitx
% some guidelines https://physics.nist.gov/cuu/Units/checklist.html
\sisetup{
  tight-spacing  = true
  ,detect-family = true
  ,detect-mode   = true
  ,binary-units  = true    % support for MB, GB, etc.
  ,range-units   = single  % "3% to 5%" -> "3 to 5%"
  ,range-phrase  = --      % "3 to 5%"  -> "3--5%"
}

% pkg: babel, hyperref
\addto\extrasenglish{%
  \renewcommand{\chapterautorefname}{Chapter}
  \renewcommand{\sectionautorefname}{Section}
  \renewcommand{\subsectionautorefname}{Section}
  \renewcommand{\subsubsectionautorefname}{Section}
}

% pkg: url
\renewcommand{\UrlFont}{\footnotesize\tt}

% our custom commands
\renewcommand{\ttdefault}{cmtt} % use computer modern for teletype

%%% draft mode / toggle commands %%%
\usepackage{etoolbox}
\newtoggle{draft}
\settoggle{draft}{true} % change toggle for draft or final versions

\iftoggle{draft}{
  % if 'draft' toggle is true
  \overfullrule=10pt                       % highlight overfull hboxes
}{
  % if 'draft' toggle is false
  \PassOptionsToPackage{final}{showlabels} % hide labels on figures, etc
}

% if you're including existing papers into your thesis, it helps to put
% content behind a toggle (or conditional) so you only have to maintain
% and keep consistency on one copy. see "introduction.tex".
\newtoggle{thesis}
\settoggle{thesis}{true}

\usepackage[inline]{showlabels}
\renewcommand{\showlabelfont}{\sffamily \color{blue}}
\renewcommand{\showlabelsetlabel}[1]{\efbox{\showlabelfont #1}}
%%%%%%%%%%%%%%%%%%%%%%%%%%%%%%%%%%%%%%%%%%%%%%%%%%%%%%%%%%%%%%%%%%%%%%%%

%% sboysel customization %%%%%%%%%%%%%%%%%%%%%%%%%%%%%%%%%%%%%%%%%%%%%%%

%\usepackage[margin=1in]{geometry}
%\usepackage{lipsum}
%\usepackage{abstract}
\usepackage{amsfonts}
%\usepackage{amssymb}
\usepackage{amsthm}
%\usepackage{amsmath}
\usepackage{bm}
\usepackage{bbm}
\usepackage{breqn}
\usepackage{epigraph}
\usepackage{tikz}
\usepackage{tikz-network}
\usepackage{pgfplots}
\pgfplotsset{compat=1.18}
\usepackage{xcolor}
\usepackage{setspace}
\usepackage{enumitem}
\usepackage[mathscr]{euscript}
\usepackage{subcaption}
\usepackage{graphicx}
\usepackage{subfiles}
%\usepackage[round]{natbib}
%\usepackage{hyperref}
%\hypersetup{
%    colorlinks=true,
%    linkcolor=blue,
%    citecolor=blue,
%    anchorcolor=blue,
%    filecolor=magenta,
%    urlcolor=cyan,
%    pdftitle={Digital Supply Chains [Boysel 2022]}
%}
%\setlength{\parindent}{0pt}
%\setlength{\parskip}{1em}
%\usepackage{titlesec}
%\titlespacing\section{0pt}{12pt plus 4pt minus 2pt}{0pt plus 2pt minus 2pt}
%\titlespacing\subsection{0pt}{12pt plus 4pt minus 2pt}{0pt plus 2pt minus 2pt}
%\titlespacing\subsubsection{0pt}{12pt plus 4pt minus 2pt}{0pt plus 2pt minus 2pt}
%\usepackage{fancyhdr}
\usepackage{caption}
%%\captionsetup[table]{skip=10pt}
\usepackage{tabularx}
\usepackage{threeparttable}
\usepackage{array}
\newcolumntype{Y}{>{\centering\arraybackslash}X}
\newcolumntype{d}[1]{D{.}{.}{#1}} 
\usepackage{booktabs}
%\usepackage{lastpage}

%%% Theorem Environments
\newtheorem{theorem}{Theorem}[section]
%\newtheorem{acknowledgement}{Acknowledgement}
\newtheorem{assumption}{Assumption}
%\newtheorem{algorithm}[theorem]{Algorithm}
%\newtheorem{axiom}[theorem]{Axiom}
%\newtheorem{case}[theorem]{Case}
%\newtheorem{claim}[theorem]{Claim}
%\newtheorem{conclusion}[theorem]{Conclusion}
%%\newtheorem{condition}[theorem]{Condition}
%\newtheorem{conjecture}[theorem]{Conjecture}
%\newtheorem{corollary}[theorem]{Corollary}
%\newtheorem{criterion}[theorem]{Criterion}
%\newtheorem{definition}[theorem]{Definition}
\newtheorem{example}[theorem]{Example}
%\newtheorem{exercise}[theorem]{Exercise}
%\newtheorem{lemma}[theorem]{Lemma}
%\newtheorem{notation}[theorem]{Notation}
%\newtheorem{problem}[theorem]{Problem}
\newtheorem{proposition}{Proposition}
\newtheorem*{remark}{Remark}
%\newtheorem{solution}[theorem]{Solution}
%\newtheorem{summary}[theorem]{Summary}

%%% New Macros
\renewcommand\qed{$\blacksquare$}
\newcommand{\st}{\, s.t. \,}
\newcommand{\grad}{\bigtriangledown}
\DeclareMathOperator*{\argmax}{arg\,max}
\DeclareMathOperator*{\argmin}{arg\,min}
\DeclareMathSymbol{\shortminus}{\mathbin}{AMSa}{"39}
\newcommand{\Cov}{\mathrm{Cov}}

%% URLs %%%
\urldef\urllibiosourcerank\url{https://docs.libraries.io/overview.html#sourcerank}

\newif\ifdissertation
\dissertationtrue     % (or \dissertationfalse for the standalone)

%% chapter appendices %%%
%%%% source: https://tex.stackexchange.com/q/120716/188684

\usepackage[titles]{tocloft}
%\setcounter{tocdepth}{2}
\usepackage{appendix}
\usepackage{chngcntr}
\usepackage{etoolbox}

% Start of subappendices environment
\AtBeginEnvironment{subappendices}{%
\chapter*{Appendices}
\addcontentsline{toc}{section}{Appendices}
\counterwithin{figure}{section}
\counterwithin{table}{section}
}
% End of subappendices environment
\AtEndEnvironment{subappendices}{%
\counterwithout{figure}{section}
\counterwithout{table}{section}
}

%% Bibliography %%%
\addbibresource{../intro/references.bib}
\addbibresource{../pub-goods/references.bib}
\addbibresource{../formation/references.bib}

%% Figures %%%
\graphicspath{{../pub-goods/figures/}{../formation/figures/}}

%%%%%%%%%%%%%%%%%%%%%%%%%%%%%%%%%%%%%%%%%%%%%%%%%%%%%%%%%%%%%%%%%%%%%%%%

%%% front matter %%%%%%%%%%%%%%%%%%%%%%%%%%%%%%%%%%%%%%%%%%%%%%%%%%%%%%%
\begin{document}

% title should be all caps
\title{SUSTAINING OPEN SOURCE SOFTWARE PRODUCTION: AN EMPIRICAL ANALYSIS THROUGH
THE LENS OF MICROECONOMICS}

% use your full name!
% https://cs.stanford.edu/~knuth/news19.html
% "Let's celebrate everybody's full names"
\author{Samuel Jospeh Boysel}

% major should be all caps
\majorfield{ECONOMICS}

% date should be May, August, or December (when degrees are conferred)
\submitdate{December 2022}

%%% preface %%%%%%%%%%%%%%%%%%%%%%%%%%%%%%%%%%%%%%%%%%%%%%%%%%%%%%%%%%%%
\begin{preface}
  \prefacesection{Dedication}
  To my mother, father, Claire, and Jake, who give my work meaning.


  \prefacesection{Acknowledgements}
  First and foremost, I am deeply indebted to my advisor Matthew Kahn, whose
support and guidance throughout my academic career has been second to none. I am
also grateful for the detailed feedback on this work provided by my dissertation
committee members David Kempke, Paulina Oliva, and Robert Metcalfe.  I
appreciate the work my committee has done in providing high-level insight,
catching errors or problems, and suggesting strategies to achieve my research
objectives. Additionally, I thank Shane Greenstein for his thorough review of
early iterations of my work. Moreover, conversations with Professors Cheng
Hsiao, Jeff Weaver, Vittorio Bassi, Monica Morlacco, and Geert Ridder helped
refine my thought process and methodology. I'd like to also thank the unyielding
support of Young Miller and Annie Le, whose tireless efforts within the USC
economics department too often go without proper acknowledgement.

Finally, comments from colleagues Rajat Kochar, Ruozi Song, Nicolas Roig, Karim
Fajury, Thomas Ash, Islamul Haque, Amy Mahler, Yue Fang, Liying Yang, Xiongfei
Li, Taraq Khan, and Amanda Ang have all helped shaped my research in positive
directions. More importantly, I cherish their friendship in ways that transcend
research.



  {
  \hypersetup{hidelinks} % color all links black in the preface
  \tableofcontents
  % fix numwidth for subappendix figs/tables
  \makeatletter
    \def\l@table{\@dottedtocline{1}{1em}{4em}}
    \def\l@figure{\@dottedtocline{1}{1em}{4em}}
    \onehalfspacing
  \makeatother
  \listoftables
  \listoffigures
  }

  \prefacesection{Abstract}
  %%\lipsum[2-3]

In this manuscript, we explore microeconomic behavior shaping the production of
open source software (OSS).  We fortify our analysis with economic structure to
guide our narrative and assess our hypotheses empirically, filling important
gaps in the literature on the supply side of markets for OSS goods.  Our
motivation is rooted in a desire to better understand how various microeconomic
phenomena influence sustained development of widely used OSS infrastructure.
Following an introduction in Chapter~\ref{ch:introduction}, our contribution is
divided into two distinct chapters.

In Chapter~\ref{ch:pg}, we examine the extent to which peer effects influence
the private provision of public goods.  In the case of public information goods,
peer contribution may facilitate or otherwise incentivize further contribution
from others, effectively subsidizing private provision.  We first utilize a
reduced form approach to derive causal estimates of net peer effects in public
goods contribution by exploiting a peers-of-peers identification strategy.  We
next develop a structural model of peer-influenced public good provision that
both (1) separates extensive and intensive margin contribution decisions and (2)
decomposes contribution into marginal private benefits and costs.  We apply
these methodologies using a sample of peer contribution histories for 2,287 OSS
projects hosted on the GitHub collaboration platform.  Both reduced form and
structural approaches suggest peer effects are much stronger along the extensive
margin than the intensive margin.  Contemporaneous intensive margin effects,
while heterogenous across time and projects, are small and centered around zero,
suggesting that strategic complementarity and substitution in peer contribution
likely offset one another.  Our counterfactual analysis suggests (extensive
margin) peer effects account for nearly 56\% of cumulative aggregate
contribution for our sample, which translates to a value-added of 1-1.5 million
software developer labor hours.  These results support the notion that OSS is
largely developed by disproportionate efforts from smaller groups of dedicated
core maintainers, who integrate incremental contributions from the wider
community, and casts doubt on the potential for peer effects alone to deliver
sustained maintenance labor to individual projects.

In Chapter~\ref{ch:dsc}, we turn our attention to the formation of software
dependency networks.  Developers of software projects can leverage the
functionality of existing open source projects.  This practice can potentially
lower the cost of development albeit at the inherent risk of relying on external
components. A ``downstream'' project maintainer can choose to ``import''
elements of an ``upstream'' project to outsource functionality, but is uncertain
how future changes in this dependency project may expose her own project to
software faults or vulnerabilities. Software dependency networks therefore
represent a ``digital supply chain'', an ecosystem of interdependent public
goods that confer an intricate set of both positive and negative externalities
for project maintainers and end users.  Focusing on microeconomic fundamentals
of the dependency management problem faced by the risk averse project
maintainer, we use both reduced form and structural approaches to study how
dependency networks create value, what forces shape their formation, and how
individual behavior can influence the robustness of equilibrium network
structure.  We use a sample of open source software projects from the Node.js
JavaScript packaging ecosystem for which contribution and dependency formation
decisions are observed in real-time. Finally, we consider several policy
interventions that can improve equilibrium welfare.  In particular, we find that
removing less that 1\% of core projects can reduce aggregate project quality by
more than 5\% for the remaining peers.


\end{preface}

%%% introduction %%%%%%%%%%%%%%%%%%%%%%%%%%%%%%%%%%%%%%%%%%%%%%%%%%%%%%%
\chapter{Introduction}\label{ch:introduction}

%\graphicspath{}
\input{../intro/intro.tex}

%%% chapter %%%%%%%%%%%%%%%%%%%%%%%%%%%%%%%%%%%%%%%%%%%%%%%%%%%%%%%%%%%%

% if you're "stapling" together papers, it's easy to include your paper
% directory by way of symlinks, or copying the entire paper as a
% subdirectory.
%
% for example, if your paper directory looks like the following:
%
%   foobar/          - top level paper directory
%   foobar/fig/      - where all graphics and figures live
%   foobar/paper.bib - bibliography
%   foobar/paper.tex - monolithic .tex file for paper
%
% then you might use the folloiwng:
%
%   \graphicspath{foobar/fig}
%   \addbibresource{foobar/paper.bib}
%   \documentclass[11pt]{report}
\usepackage[
  dissertation
 ,final
 ,raggedbottom
%,tocbold        % uncomment to enable bold chapter titles in the ToC
                 %
                 % the style guidelines state that page numbers in the
                 % ToC should not be bold, but leave it up to the author
                 % and specific department guidelines as to how the
                 % chapter (or other section) titles should be typeset.
]{USCthesis}

% guidelines for manuscript formatting: https://graduateschool.usc.edu/wp-content/uploads/2020/11/Manuscript_Formatting_and_Documentation_Styles.pdf

%% our customizations %%%%%%%%%%%%%%%%%%%%%%%%%%%%%%%%%%%%%%%%%%%%%%%%%%
\usepackage[export]{adjustbox} % for frame option in \includegraphics
\usepackage{amsmath}
\usepackage{amssymb}
\usepackage{array}
\usepackage[utf8]{inputenc} % load inputenc before csquotes
\usepackage[english]{babel}
\usepackage[
  backend     = biber,
  citestyle   = authoryear,
  doi         = true,
  hyperref    = true,
  maxbibnames = 99,
  natbib      = true,
  sortlocale  = en_US,
  style       = authoryear,
]{biblatex}
\usepackage{booktabs}
\usepackage{color, colortbl}
\usepackage{csquotes}
\usepackage{efbox}
\usepackage{enumitem}
\usepackage[shortcuts]{extdash} % use `\-/' to hyphenate words/phrases that have a dash in them
% \usepackage[tt=false]{libertine} % libertine's \ttfamily isn't that great
\usepackage[T1]{fontenc} % load fonts before fontenc
%\usepackage[symbol]{footmisc}
\usepackage{footmisc}
\usepackage[
  showframe = false,% draw a border around textwidth
  pass      = true, % force 8.5"x11" pagesize
]{geometry}
\usepackage{graphicx}
%\usepackage[notquote]{hanging} % enables negative indents in paragraphs
\usepackage{hyphenat}
\usepackage{ifthen}
\usepackage{lipsum}
\usepackage{multirow}
\usepackage{parnotes}
\usepackage{pdflscape} % rotate some pages in an {landscape} environment
\usepackage{pifont}
\usepackage{ragged2e}
\usepackage{seqsplit}
\usepackage{siunitx}
\usepackage{subcaption}
\usepackage{tabularx}
\usepackage{xcolor}
\usepackage{xspace}
\usepackage{url}

\usepackage[
  bookmarks     = true,
  breaklinks    = true,
  colorlinks    = true,
  hypertexnames = false,
  pdfpagelabels = false,
  citecolor     = {blue!80!black},
  linkcolor     = {blue!80!black},
  urlcolor      = {blue!80!black},
]{hyperref} % load hyperref as the last package

% pkg: biblatex
\setlength\bibitemsep{0.5\baselineskip}                 % add a line between entries
\AtEveryBibitem{\iffieldundef{doi}{}{\clearfield{url}}} % if DOI, hide URL

% \addbibresource{paper.bib}

% pkg: siunitx
% some guidelines https://physics.nist.gov/cuu/Units/checklist.html
\sisetup{
  tight-spacing  = true
  ,detect-family = true
  ,detect-mode   = true
  ,binary-units  = true    % support for MB, GB, etc.
  ,range-units   = single  % "3% to 5%" -> "3 to 5%"
  ,range-phrase  = --      % "3 to 5%"  -> "3--5%"
}

% pkg: babel, hyperref
\addto\extrasenglish{%
  \renewcommand{\chapterautorefname}{Chapter}
  \renewcommand{\sectionautorefname}{Section}
  \renewcommand{\subsectionautorefname}{Section}
  \renewcommand{\subsubsectionautorefname}{Section}
}

% pkg: url
\renewcommand{\UrlFont}{\footnotesize\tt}

% our custom commands
\renewcommand{\ttdefault}{cmtt} % use computer modern for teletype

%%% draft mode / toggle commands %%%
\usepackage{etoolbox}
\newtoggle{draft}
\settoggle{draft}{true} % change toggle for draft or final versions

\iftoggle{draft}{
  % if 'draft' toggle is true
  \overfullrule=10pt                       % highlight overfull hboxes
}{
  % if 'draft' toggle is false
  \PassOptionsToPackage{final}{showlabels} % hide labels on figures, etc
}

% if you're including existing papers into your thesis, it helps to put
% content behind a toggle (or conditional) so you only have to maintain
% and keep consistency on one copy. see "introduction.tex".
\newtoggle{thesis}
\settoggle{thesis}{true}

\usepackage[inline]{showlabels}
\renewcommand{\showlabelfont}{\sffamily \color{blue}}
\renewcommand{\showlabelsetlabel}[1]{\efbox{\showlabelfont #1}}
%%%%%%%%%%%%%%%%%%%%%%%%%%%%%%%%%%%%%%%%%%%%%%%%%%%%%%%%%%%%%%%%%%%%%%%%

%% sboysel customization %%%%%%%%%%%%%%%%%%%%%%%%%%%%%%%%%%%%%%%%%%%%%%%

%\usepackage[margin=1in]{geometry}
%\usepackage{lipsum}
%\usepackage{abstract}
\usepackage{amsfonts}
%\usepackage{amssymb}
\usepackage{amsthm}
%\usepackage{amsmath}
\usepackage{bm}
\usepackage{bbm}
\usepackage{breqn}
\usepackage{epigraph}
\usepackage{tikz}
\usepackage{tikz-network}
\usepackage{pgfplots}
\pgfplotsset{compat=1.18}
\usepackage{xcolor}
\usepackage{setspace}
\usepackage{enumitem}
\usepackage[mathscr]{euscript}
\usepackage{subcaption}
\usepackage{graphicx}
\usepackage{subfiles}
%\usepackage[round]{natbib}
%\usepackage{hyperref}
%\hypersetup{
%    colorlinks=true,
%    linkcolor=blue,
%    citecolor=blue,
%    anchorcolor=blue,
%    filecolor=magenta,
%    urlcolor=cyan,
%    pdftitle={Digital Supply Chains [Boysel 2022]}
%}
%\setlength{\parindent}{0pt}
%\setlength{\parskip}{1em}
%\usepackage{titlesec}
%\titlespacing\section{0pt}{12pt plus 4pt minus 2pt}{0pt plus 2pt minus 2pt}
%\titlespacing\subsection{0pt}{12pt plus 4pt minus 2pt}{0pt plus 2pt minus 2pt}
%\titlespacing\subsubsection{0pt}{12pt plus 4pt minus 2pt}{0pt plus 2pt minus 2pt}
%\usepackage{fancyhdr}
\usepackage{caption}
%%\captionsetup[table]{skip=10pt}
\usepackage{tabularx}
\usepackage{threeparttable}
\usepackage{array}
\newcolumntype{Y}{>{\centering\arraybackslash}X}
\newcolumntype{d}[1]{D{.}{.}{#1}} 
\usepackage{booktabs}
%\usepackage{lastpage}

%%% Theorem Environments
\newtheorem{theorem}{Theorem}[section]
%\newtheorem{acknowledgement}{Acknowledgement}
\newtheorem{assumption}{Assumption}
%\newtheorem{algorithm}[theorem]{Algorithm}
%\newtheorem{axiom}[theorem]{Axiom}
%\newtheorem{case}[theorem]{Case}
%\newtheorem{claim}[theorem]{Claim}
%\newtheorem{conclusion}[theorem]{Conclusion}
%%\newtheorem{condition}[theorem]{Condition}
%\newtheorem{conjecture}[theorem]{Conjecture}
%\newtheorem{corollary}[theorem]{Corollary}
%\newtheorem{criterion}[theorem]{Criterion}
%\newtheorem{definition}[theorem]{Definition}
\newtheorem{example}[theorem]{Example}
%\newtheorem{exercise}[theorem]{Exercise}
%\newtheorem{lemma}[theorem]{Lemma}
%\newtheorem{notation}[theorem]{Notation}
%\newtheorem{problem}[theorem]{Problem}
\newtheorem{proposition}{Proposition}
\newtheorem*{remark}{Remark}
%\newtheorem{solution}[theorem]{Solution}
%\newtheorem{summary}[theorem]{Summary}

%%% New Macros
\renewcommand\qed{$\blacksquare$}
\newcommand{\st}{\, s.t. \,}
\newcommand{\grad}{\bigtriangledown}
\DeclareMathOperator*{\argmax}{arg\,max}
\DeclareMathOperator*{\argmin}{arg\,min}
\DeclareMathSymbol{\shortminus}{\mathbin}{AMSa}{"39}
\newcommand{\Cov}{\mathrm{Cov}}

%% URLs %%%
\urldef\urllibiosourcerank\url{https://docs.libraries.io/overview.html#sourcerank}

\newif\ifdissertation
\dissertationtrue     % (or \dissertationfalse for the standalone)

%% chapter appendices %%%
%%%% source: https://tex.stackexchange.com/q/120716/188684

\usepackage[titles]{tocloft}
%\setcounter{tocdepth}{2}
\usepackage{appendix}
\usepackage{chngcntr}
\usepackage{etoolbox}

% Start of subappendices environment
\AtBeginEnvironment{subappendices}{%
\chapter*{Appendices}
\addcontentsline{toc}{section}{Appendices}
\counterwithin{figure}{section}
\counterwithin{table}{section}
}
% End of subappendices environment
\AtEndEnvironment{subappendices}{%
\counterwithout{figure}{section}
\counterwithout{table}{section}
}

%% Bibliography %%%
\addbibresource{../intro/references.bib}
\addbibresource{../pub-goods/references.bib}
\addbibresource{../formation/references.bib}

%% Figures %%%
\graphicspath{{../pub-goods/figures/}{../formation/figures/}}

%%%%%%%%%%%%%%%%%%%%%%%%%%%%%%%%%%%%%%%%%%%%%%%%%%%%%%%%%%%%%%%%%%%%%%%%

%%% front matter %%%%%%%%%%%%%%%%%%%%%%%%%%%%%%%%%%%%%%%%%%%%%%%%%%%%%%%
\begin{document}

% title should be all caps
\title{SUSTAINING OPEN SOURCE SOFTWARE PRODUCTION: AN EMPIRICAL ANALYSIS THROUGH
THE LENS OF MICROECONOMICS}

% use your full name!
% https://cs.stanford.edu/~knuth/news19.html
% "Let's celebrate everybody's full names"
\author{Samuel Jospeh Boysel}

% major should be all caps
\majorfield{ECONOMICS}

% date should be May, August, or December (when degrees are conferred)
\submitdate{December 2022}

%%% preface %%%%%%%%%%%%%%%%%%%%%%%%%%%%%%%%%%%%%%%%%%%%%%%%%%%%%%%%%%%%
\begin{preface}
  \prefacesection{Dedication}
  \input{dedication.tex}

  \prefacesection{Acknowledgements}
  \input{acknowledgements.tex}

  {
  \hypersetup{hidelinks} % color all links black in the preface
  \tableofcontents
  % fix numwidth for subappendix figs/tables
  \makeatletter
    \def\l@table{\@dottedtocline{1}{1em}{4em}}
    \def\l@figure{\@dottedtocline{1}{1em}{4em}}
    \onehalfspacing
  \makeatother
  \listoftables
  \listoffigures
  }

  \prefacesection{Abstract}
  \input{abstract.tex}
\end{preface}

%%% introduction %%%%%%%%%%%%%%%%%%%%%%%%%%%%%%%%%%%%%%%%%%%%%%%%%%%%%%%
\chapter{Introduction}\label{ch:introduction}

%\graphicspath{}
\input{../intro/intro.tex}

%%% chapter %%%%%%%%%%%%%%%%%%%%%%%%%%%%%%%%%%%%%%%%%%%%%%%%%%%%%%%%%%%%

% if you're "stapling" together papers, it's easy to include your paper
% directory by way of symlinks, or copying the entire paper as a
% subdirectory.
%
% for example, if your paper directory looks like the following:
%
%   foobar/          - top level paper directory
%   foobar/fig/      - where all graphics and figures live
%   foobar/paper.bib - bibliography
%   foobar/paper.tex - monolithic .tex file for paper
%
% then you might use the folloiwng:
%
%   \graphicspath{foobar/fig}
%   \addbibresource{foobar/paper.bib}
%   \input{foobar/paper.tex}
%
% note that you'll have to modify the input file to make sure that the
% preamble (\documentclass, etc.) isn't included. to make your life
% easier, you could use some TeX conditionals to make it seamless.
%
% this requires some planning, but enables you to edit the individual
% paper and thesis chapter without tracking and porting changes between
% multiple directories and repositories:
%
% for example, at the beginning of foobar/paper.tex (before
% \documentclass):
%
%   \newif\ifdissertation
%   \dissertationtrue      % (or \dissertationfalse for the standalone)
%
%   \ifdissertation
%   \else
%   \documentclass...
%   \fi
%
%   \ifdissertation
%   \else
%   \begin{document}
%   \fi
%
%   [...paper content here...]
%
%   \ifdissertation
%   \else
%   \end{document}
%   \fi

%%% chapters: lorem ipsum %%%%%%%%%%%%%%%%%%%%%%%%%%%%%%%%%%%%%%%%%%%%%%

% The following text is to test ToC alignment:
% - of extremely long chapter, section, subsection, and
%   subsubsection titles
% - when chapter numbers are double digits

% \chapter{This is a very long title which will take up more than one line
% lorem ipsum dolor sit amet, consectetur adipiscing elit, sed do eiusmod
% tempor incididunt ut labore et dolore magna aliqua. Ultrices vitae
% auctor eu augue ut lectus arcu. Enim nunc faucibus a pellentesque sit
% amet porttitor eget. Consequat mauris nunc congue nisi vitae.}
%   \label{ch:long-title}

% \section{Ut enim ad minim veniam, quis nostrud exercitation ullamco
% laboris nisi ut aliquip ex ea commodo consequat}

% \subsection{Duis aute irure dolor in reprehenderit in voluptate velit
% esse cillum dolore eu fugiat nulla pariatur. Excepteur sint occaecat
% cupidatat non proident, sunt in culpa qui officia deserunt mollit anim
% id est laborum.}

% \subsubsection{Etiam erat velit scelerisque in dictum non. Sit amet
% justo donec enim diam. Amet justo donec enim diam. Metus vulputate eu
% scelerisque felis imperdiet proin. In nulla posuere sollicitudin aliquam
% ultrices. Turpis in eu mi bibendum.}

% \chapter{Lorem Ipsum}
% \chapter{Lorem Ipsum}
% \chapter{Lorem Ipsum}
% \chapter{Lorem Ipsum}
% \chapter{Lorem Ipsum}
% \chapter{Lorem Ipsum}
% \chapter{Lorem Ipsum}
% \chapter{Lorem Ipsum}
% \chapter{Etiam erat velit scelerisque in dictum non. Sit amet
% justo donec enim diam. Amet justo donec enim diam. Metus vulputate eu
% scelerisque felis imperdiet proin. In nulla posuere sollicitudin aliquam
% ultrices. Turpis in eu mi bibendum.}

% \begin{table}
% \centering
% \begin{tabular}{lS}
% \toprule
% $x$      & \textbf{value} \\
% \midrule
% a        & 1.23           \\
% b        & 3.456          \\
% c        & 100.0002       \\
% d        & 12345.0        \\
% \bottomrule
% \end{tabular}
% \caption[In hendrerit gravida rutrum quisque non tellus orci ac. Iaculis
%         urna id volutpat lacus laoreet non curabitur gravida arcu. Mauris
%         ultrices eros in cursus turpis massa. Sed tempus urna et pharetra
%         pharetra massa massa. Eget sit amet tellus cras adipiscing enim eu
%         turpis egestas. Morbi blandit cursus risus at ultrices.]
%         {In hendrerit gravida rutrum quisque non tellus orci ac. Iaculis
%         urna id volutpat lacus laoreet non curabitur gravida arcu.
%         Mauris ultrices eros in cursus turpis massa. Sed tempus urna et
%         pharetra pharetra massa massa. Eget sit amet tellus cras
%         adipiscing enim eu turpis egestas. Morbi blandit cursus risus at
%         ultrices.}
% \label{tbl:example-2}
% \end{table}

%%% chapter 1 %%%%%%%%%%%%%%%%%%%%%%%%%%%%%%%%%%%%%%%%%%%%%%%%%%%%%%%%%%%%%

\chapter{Quid Pro Code: Peer Effects and Productivity in Open Source Software}\label{ch:pg}

\section{Introduction}\label{sec:pg-intro}
\subfile{../pub-goods/intro/intro.tex}

\section{Background}\label{sec:pg-background}
\subfile{../pub-goods/background/background.tex}

\section{Literature}\label{sec:pg-lit}
\subfile{../pub-goods/lit/lit.tex}

\section{Data}\label{sec:pg-data}
\subfile{../pub-goods/data/data.tex}

\section{Reduced Form}\label{sec:pg-reduced}
\subfile{../pub-goods/reduced/reduced.tex}

\section{Structural Model}\label{sec:pg-structural}
\subfile{../pub-goods/struct/struct.tex}

\section{Discussion}\label{sec:pg-discussion}
\subfile{../pub-goods/discussion/discussion.tex}

\clearpage
\begin{subappendices}
\section{Tables}\label{app:pg-tables}
\subfile{../pub-goods/appendix/tables/tables.tex}

\clearpage
\section{Figures}\label{app:pg-figures}
\subfile{../pub-goods/appendix/figures/figures.tex}

\clearpage
\section{Data Details}\label{app:pg-data-app}
\subfile{../pub-goods/appendix/data/data.tex}

\clearpage
% \onehalfspacing
\section{Additional Reduced Form Results}\label{app:pg-reduced-details}
\subfile{../pub-goods/appendix/reduced-details/reduced-details.tex}

\clearpage
% \onehalfspacing
\section{Structural Estimation Details}\label{app:pg-estim-details}
\subfile{../pub-goods/appendix/estim/estim.tex}

\end{subappendices}

%%% chapter 2 %%%%%%%%%%%%%%%%%%%%%%%%%%%%%%%%%%%%%%%%%%%%%%%%%%%%%%%%%%%%%

\chapter{No Free Lunch For Programmers: Digital Supply Chains and the Economics of Software Dependency Management}\label{ch:dsc}

\section{Introduction}\label{sec:dsc-intro}
\subfile{../formation/intro/intro}

\section{Literature}\label{sec:dsc-lit}
\subfile{../formation/lit/lit}

\section{Framework}\label{sec:dsc-framework}
\subfile{../formation/framework/framework}

\section{Data}\label{sec:dsc-data}
\subfile{../formation/data/data}

\section{Reduced Form}\label{sec:dsc-reduced}
\subfile{../formation/reduced/reduced}

\section{Structural Approach}\label{sec:dsc-structural}
\subfile{../formation/structural/structural}

\section{Counterfactual Analysis}\label{sec:dsc-counterfactual}
\subfile{../formation/counterfactual/counterfactual}

\section{Discussion}\label{sec:dsc-discussion}
\subfile{../formation/discussion/discussion}

\clearpage
\begin{subappendices}
\section{Figures}\label{app:dsc-figures}
\subfile{../formation/appendix/figures/figures.tex}

\clearpage
\section{Tables}\label{app:dsc-tables}
\subfile{../formation/appendix/tables/tables.tex}

\clearpage
\section{Mathematical Details}\label{app:dsc-math}
\subfile{../formation/appendix/math/math.tex}

\clearpage
\section{Estimation Details}\label{app:dsc-estimation}
\subfile{../formation/appendix/estimation/estimation.tex}

\end{subappendices}

% %%% conclusions %%%%%%%%%%%%%%%%%%%%%%%%%%%%%%%%%%%%%%%%%%%%%%%%%%%%%%%%
% \chapter{Conclusions}
%   \label{ch:conclusions}

% \graphicspath{}
% \input{conclusions}

%%% bibliography %%%%%%%%%%%%%%%%%%%%%%%%%%%%%%%%%%%%%%%%%%%%%%%%%%%%%%%
%
%  \printbibliography in biblatex is great, but doesn't allow for the
%  greatest customization, so we'll use the package biblatex + biber
%  backend to meet some requirements:
%
%  * bibliography should be an un-numbered chapter, and still have a
%    pdfbookmark and a line in the table of contents
%
%  * bibliography contents should be singlespace, and optionally a smaller
%    font
%
%  * first line of this "chapter" should be in the same spot as the first
%    line of preface sections (e.g., acknowledgement)
%
%  * we use \raggedright so things like URLs and DOIs aren't stretched out.
%
\clearpage
\chapter*{Bibliography}
\addcontentsline{toc}{chapter}{Bibliography}

\begin{singlespace}
  % increase penalty such that we don't break entries over pages
  % source: https://tex.stackexchange.com/a/43275
  \patchcmd{\bibsetup}{\interlinepenalty=5000}{\interlinepenalty=10000}{}{}

  % reduce spacing between each bibentry
  \setlength\bibitemsep{0.9\baselineskip}

  % don't justify-align entries: this prevents stretching out each line
  \raggedright
  \printbibliography[
    heading = none
  ]
\end{singlespace}

\end{document}

%
% note that you'll have to modify the input file to make sure that the
% preamble (\documentclass, etc.) isn't included. to make your life
% easier, you could use some TeX conditionals to make it seamless.
%
% this requires some planning, but enables you to edit the individual
% paper and thesis chapter without tracking and porting changes between
% multiple directories and repositories:
%
% for example, at the beginning of foobar/paper.tex (before
% \documentclass):
%
%   \newif\ifdissertation
%   \dissertationtrue      % (or \dissertationfalse for the standalone)
%
%   \ifdissertation
%   \else
%   \documentclass...
%   \fi
%
%   \ifdissertation
%   \else
%   \begin{document}
%   \fi
%
%   [...paper content here...]
%
%   \ifdissertation
%   \else
%   \end{document}
%   \fi

%%% chapters: lorem ipsum %%%%%%%%%%%%%%%%%%%%%%%%%%%%%%%%%%%%%%%%%%%%%%

% The following text is to test ToC alignment:
% - of extremely long chapter, section, subsection, and
%   subsubsection titles
% - when chapter numbers are double digits

% \chapter{This is a very long title which will take up more than one line
% lorem ipsum dolor sit amet, consectetur adipiscing elit, sed do eiusmod
% tempor incididunt ut labore et dolore magna aliqua. Ultrices vitae
% auctor eu augue ut lectus arcu. Enim nunc faucibus a pellentesque sit
% amet porttitor eget. Consequat mauris nunc congue nisi vitae.}
%   \label{ch:long-title}

% \section{Ut enim ad minim veniam, quis nostrud exercitation ullamco
% laboris nisi ut aliquip ex ea commodo consequat}

% \subsection{Duis aute irure dolor in reprehenderit in voluptate velit
% esse cillum dolore eu fugiat nulla pariatur. Excepteur sint occaecat
% cupidatat non proident, sunt in culpa qui officia deserunt mollit anim
% id est laborum.}

% \subsubsection{Etiam erat velit scelerisque in dictum non. Sit amet
% justo donec enim diam. Amet justo donec enim diam. Metus vulputate eu
% scelerisque felis imperdiet proin. In nulla posuere sollicitudin aliquam
% ultrices. Turpis in eu mi bibendum.}

% \chapter{Lorem Ipsum}
% \chapter{Lorem Ipsum}
% \chapter{Lorem Ipsum}
% \chapter{Lorem Ipsum}
% \chapter{Lorem Ipsum}
% \chapter{Lorem Ipsum}
% \chapter{Lorem Ipsum}
% \chapter{Lorem Ipsum}
% \chapter{Etiam erat velit scelerisque in dictum non. Sit amet
% justo donec enim diam. Amet justo donec enim diam. Metus vulputate eu
% scelerisque felis imperdiet proin. In nulla posuere sollicitudin aliquam
% ultrices. Turpis in eu mi bibendum.}

% \begin{table}
% \centering
% \begin{tabular}{lS}
% \toprule
% $x$      & \textbf{value} \\
% \midrule
% a        & 1.23           \\
% b        & 3.456          \\
% c        & 100.0002       \\
% d        & 12345.0        \\
% \bottomrule
% \end{tabular}
% \caption[In hendrerit gravida rutrum quisque non tellus orci ac. Iaculis
%         urna id volutpat lacus laoreet non curabitur gravida arcu. Mauris
%         ultrices eros in cursus turpis massa. Sed tempus urna et pharetra
%         pharetra massa massa. Eget sit amet tellus cras adipiscing enim eu
%         turpis egestas. Morbi blandit cursus risus at ultrices.]
%         {In hendrerit gravida rutrum quisque non tellus orci ac. Iaculis
%         urna id volutpat lacus laoreet non curabitur gravida arcu.
%         Mauris ultrices eros in cursus turpis massa. Sed tempus urna et
%         pharetra pharetra massa massa. Eget sit amet tellus cras
%         adipiscing enim eu turpis egestas. Morbi blandit cursus risus at
%         ultrices.}
% \label{tbl:example-2}
% \end{table}

%%% chapter 1 %%%%%%%%%%%%%%%%%%%%%%%%%%%%%%%%%%%%%%%%%%%%%%%%%%%%%%%%%%%%%

\chapter{Quid Pro Code: Peer Effects and Productivity in Open Source Software}\label{ch:pg}

\section{Introduction}\label{sec:pg-intro}
\subfile{../pub-goods/intro/intro.tex}

\section{Background}\label{sec:pg-background}
\subfile{../pub-goods/background/background.tex}

\section{Literature}\label{sec:pg-lit}
\subfile{../pub-goods/lit/lit.tex}

\section{Data}\label{sec:pg-data}
\subfile{../pub-goods/data/data.tex}

\section{Reduced Form}\label{sec:pg-reduced}
\subfile{../pub-goods/reduced/reduced.tex}

\section{Structural Model}\label{sec:pg-structural}
\subfile{../pub-goods/struct/struct.tex}

\section{Discussion}\label{sec:pg-discussion}
\subfile{../pub-goods/discussion/discussion.tex}

\clearpage
\begin{subappendices}
\section{Tables}\label{app:pg-tables}
\subfile{../pub-goods/appendix/tables/tables.tex}

\clearpage
\section{Figures}\label{app:pg-figures}
\subfile{../pub-goods/appendix/figures/figures.tex}

\clearpage
\section{Data Details}\label{app:pg-data-app}
\subfile{../pub-goods/appendix/data/data.tex}

\clearpage
% \onehalfspacing
\section{Additional Reduced Form Results}\label{app:pg-reduced-details}
\subfile{../pub-goods/appendix/reduced-details/reduced-details.tex}

\clearpage
% \onehalfspacing
\section{Structural Estimation Details}\label{app:pg-estim-details}
\subfile{../pub-goods/appendix/estim/estim.tex}

\end{subappendices}

%%% chapter 2 %%%%%%%%%%%%%%%%%%%%%%%%%%%%%%%%%%%%%%%%%%%%%%%%%%%%%%%%%%%%%

\chapter{No Free Lunch For Programmers: Digital Supply Chains and the Economics of Software Dependency Management}\label{ch:dsc}

\section{Introduction}\label{sec:dsc-intro}
\subfile{../formation/intro/intro}

\section{Literature}\label{sec:dsc-lit}
\subfile{../formation/lit/lit}

\section{Framework}\label{sec:dsc-framework}
\subfile{../formation/framework/framework}

\section{Data}\label{sec:dsc-data}
\subfile{../formation/data/data}

\section{Reduced Form}\label{sec:dsc-reduced}
\subfile{../formation/reduced/reduced}

\section{Structural Approach}\label{sec:dsc-structural}
\subfile{../formation/structural/structural}

\section{Counterfactual Analysis}\label{sec:dsc-counterfactual}
\subfile{../formation/counterfactual/counterfactual}

\section{Discussion}\label{sec:dsc-discussion}
\subfile{../formation/discussion/discussion}

\clearpage
\begin{subappendices}
\section{Figures}\label{app:dsc-figures}
\subfile{../formation/appendix/figures/figures.tex}

\clearpage
\section{Tables}\label{app:dsc-tables}
\subfile{../formation/appendix/tables/tables.tex}

\clearpage
\section{Mathematical Details}\label{app:dsc-math}
\subfile{../formation/appendix/math/math.tex}

\clearpage
\section{Estimation Details}\label{app:dsc-estimation}
\subfile{../formation/appendix/estimation/estimation.tex}

\end{subappendices}

% %%% conclusions %%%%%%%%%%%%%%%%%%%%%%%%%%%%%%%%%%%%%%%%%%%%%%%%%%%%%%%%
% \chapter{Conclusions}
%   \label{ch:conclusions}

% \graphicspath{}
% \input{conclusions}

%%% bibliography %%%%%%%%%%%%%%%%%%%%%%%%%%%%%%%%%%%%%%%%%%%%%%%%%%%%%%%
%
%  \printbibliography in biblatex is great, but doesn't allow for the
%  greatest customization, so we'll use the package biblatex + biber
%  backend to meet some requirements:
%
%  * bibliography should be an un-numbered chapter, and still have a
%    pdfbookmark and a line in the table of contents
%
%  * bibliography contents should be singlespace, and optionally a smaller
%    font
%
%  * first line of this "chapter" should be in the same spot as the first
%    line of preface sections (e.g., acknowledgement)
%
%  * we use \raggedright so things like URLs and DOIs aren't stretched out.
%
\clearpage
\chapter*{Bibliography}
\addcontentsline{toc}{chapter}{Bibliography}

\begin{singlespace}
  % increase penalty such that we don't break entries over pages
  % source: https://tex.stackexchange.com/a/43275
  \patchcmd{\bibsetup}{\interlinepenalty=5000}{\interlinepenalty=10000}{}{}

  % reduce spacing between each bibentry
  \setlength\bibitemsep{0.9\baselineskip}

  % don't justify-align entries: this prevents stretching out each line
  \raggedright
  \printbibliography[
    heading = none
  ]
\end{singlespace}

\end{document}

%
% note that you'll have to modify the input file to make sure that the
% preamble (\documentclass, etc.) isn't included. to make your life
% easier, you could use some TeX conditionals to make it seamless.
%
% this requires some planning, but enables you to edit the individual
% paper and thesis chapter without tracking and porting changes between
% multiple directories and repositories:
%
% for example, at the beginning of foobar/paper.tex (before
% \documentclass):
%
%   \newif\ifdissertation
%   \dissertationtrue      % (or \dissertationfalse for the standalone)
%
%   \ifdissertation
%   \else
%   \documentclass...
%   \fi
%
%   \ifdissertation
%   \else
%   \begin{document}
%   \fi
%
%   [...paper content here...]
%
%   \ifdissertation
%   \else
%   \end{document}
%   \fi

%%% chapters: lorem ipsum %%%%%%%%%%%%%%%%%%%%%%%%%%%%%%%%%%%%%%%%%%%%%%

% The following text is to test ToC alignment:
% - of extremely long chapter, section, subsection, and
%   subsubsection titles
% - when chapter numbers are double digits

% \chapter{This is a very long title which will take up more than one line
% lorem ipsum dolor sit amet, consectetur adipiscing elit, sed do eiusmod
% tempor incididunt ut labore et dolore magna aliqua. Ultrices vitae
% auctor eu augue ut lectus arcu. Enim nunc faucibus a pellentesque sit
% amet porttitor eget. Consequat mauris nunc congue nisi vitae.}
%   \label{ch:long-title}

% \section{Ut enim ad minim veniam, quis nostrud exercitation ullamco
% laboris nisi ut aliquip ex ea commodo consequat}

% \subsection{Duis aute irure dolor in reprehenderit in voluptate velit
% esse cillum dolore eu fugiat nulla pariatur. Excepteur sint occaecat
% cupidatat non proident, sunt in culpa qui officia deserunt mollit anim
% id est laborum.}

% \subsubsection{Etiam erat velit scelerisque in dictum non. Sit amet
% justo donec enim diam. Amet justo donec enim diam. Metus vulputate eu
% scelerisque felis imperdiet proin. In nulla posuere sollicitudin aliquam
% ultrices. Turpis in eu mi bibendum.}

% \chapter{Lorem Ipsum}
% \chapter{Lorem Ipsum}
% \chapter{Lorem Ipsum}
% \chapter{Lorem Ipsum}
% \chapter{Lorem Ipsum}
% \chapter{Lorem Ipsum}
% \chapter{Lorem Ipsum}
% \chapter{Lorem Ipsum}
% \chapter{Etiam erat velit scelerisque in dictum non. Sit amet
% justo donec enim diam. Amet justo donec enim diam. Metus vulputate eu
% scelerisque felis imperdiet proin. In nulla posuere sollicitudin aliquam
% ultrices. Turpis in eu mi bibendum.}

% \begin{table}
% \centering
% \begin{tabular}{lS}
% \toprule
% $x$      & \textbf{value} \\
% \midrule
% a        & 1.23           \\
% b        & 3.456          \\
% c        & 100.0002       \\
% d        & 12345.0        \\
% \bottomrule
% \end{tabular}
% \caption[In hendrerit gravida rutrum quisque non tellus orci ac. Iaculis
%         urna id volutpat lacus laoreet non curabitur gravida arcu. Mauris
%         ultrices eros in cursus turpis massa. Sed tempus urna et pharetra
%         pharetra massa massa. Eget sit amet tellus cras adipiscing enim eu
%         turpis egestas. Morbi blandit cursus risus at ultrices.]
%         {In hendrerit gravida rutrum quisque non tellus orci ac. Iaculis
%         urna id volutpat lacus laoreet non curabitur gravida arcu.
%         Mauris ultrices eros in cursus turpis massa. Sed tempus urna et
%         pharetra pharetra massa massa. Eget sit amet tellus cras
%         adipiscing enim eu turpis egestas. Morbi blandit cursus risus at
%         ultrices.}
% \label{tbl:example-2}
% \end{table}

%%% chapter 1 %%%%%%%%%%%%%%%%%%%%%%%%%%%%%%%%%%%%%%%%%%%%%%%%%%%%%%%%%%%%%

\chapter{Quid Pro Code: Peer Effects and Productivity in Open Source Software}\label{ch:pg}

\section{Introduction}\label{sec:pg-intro}
\subfile{../pub-goods/intro/intro.tex}

\section{Background}\label{sec:pg-background}
\subfile{../pub-goods/background/background.tex}

\section{Literature}\label{sec:pg-lit}
\subfile{../pub-goods/lit/lit.tex}

\section{Data}\label{sec:pg-data}
\subfile{../pub-goods/data/data.tex}

\section{Reduced Form}\label{sec:pg-reduced}
\subfile{../pub-goods/reduced/reduced.tex}

\section{Structural Model}\label{sec:pg-structural}
\subfile{../pub-goods/struct/struct.tex}

\section{Discussion}\label{sec:pg-discussion}
\subfile{../pub-goods/discussion/discussion.tex}

\clearpage
\begin{subappendices}
\section{Tables}\label{app:pg-tables}
\subfile{../pub-goods/appendix/tables/tables.tex}

\clearpage
\section{Figures}\label{app:pg-figures}
\subfile{../pub-goods/appendix/figures/figures.tex}

\clearpage
\section{Data Details}\label{app:pg-data-app}
\subfile{../pub-goods/appendix/data/data.tex}

\clearpage
% \onehalfspacing
\section{Additional Reduced Form Results}\label{app:pg-reduced-details}
\subfile{../pub-goods/appendix/reduced-details/reduced-details.tex}

\clearpage
% \onehalfspacing
\section{Structural Estimation Details}\label{app:pg-estim-details}
\subfile{../pub-goods/appendix/estim/estim.tex}

\end{subappendices}

%%% chapter 2 %%%%%%%%%%%%%%%%%%%%%%%%%%%%%%%%%%%%%%%%%%%%%%%%%%%%%%%%%%%%%

\chapter{No Free Lunch For Programmers: Digital Supply Chains and the Economics of Software Dependency Management}\label{ch:dsc}

\section{Introduction}\label{sec:dsc-intro}
\subfile{../formation/intro/intro}

\section{Literature}\label{sec:dsc-lit}
\subfile{../formation/lit/lit}

\section{Framework}\label{sec:dsc-framework}
\subfile{../formation/framework/framework}

\section{Data}\label{sec:dsc-data}
\subfile{../formation/data/data}

\section{Reduced Form}\label{sec:dsc-reduced}
\subfile{../formation/reduced/reduced}

\section{Structural Approach}\label{sec:dsc-structural}
\subfile{../formation/structural/structural}

\section{Counterfactual Analysis}\label{sec:dsc-counterfactual}
\subfile{../formation/counterfactual/counterfactual}

\section{Discussion}\label{sec:dsc-discussion}
\subfile{../formation/discussion/discussion}

\clearpage
\begin{subappendices}
\section{Figures}\label{app:dsc-figures}
\subfile{../formation/appendix/figures/figures.tex}

\clearpage
\section{Tables}\label{app:dsc-tables}
\subfile{../formation/appendix/tables/tables.tex}

\clearpage
\section{Mathematical Details}\label{app:dsc-math}
\subfile{../formation/appendix/math/math.tex}

\clearpage
\section{Estimation Details}\label{app:dsc-estimation}
\subfile{../formation/appendix/estimation/estimation.tex}

\end{subappendices}

% %%% conclusions %%%%%%%%%%%%%%%%%%%%%%%%%%%%%%%%%%%%%%%%%%%%%%%%%%%%%%%%
% \chapter{Conclusions}
%   \label{ch:conclusions}

% \graphicspath{}
% \input{conclusions}

%%% bibliography %%%%%%%%%%%%%%%%%%%%%%%%%%%%%%%%%%%%%%%%%%%%%%%%%%%%%%%
%
%  \printbibliography in biblatex is great, but doesn't allow for the
%  greatest customization, so we'll use the package biblatex + biber
%  backend to meet some requirements:
%
%  * bibliography should be an un-numbered chapter, and still have a
%    pdfbookmark and a line in the table of contents
%
%  * bibliography contents should be singlespace, and optionally a smaller
%    font
%
%  * first line of this "chapter" should be in the same spot as the first
%    line of preface sections (e.g., acknowledgement)
%
%  * we use \raggedright so things like URLs and DOIs aren't stretched out.
%
\clearpage
\chapter*{Bibliography}
\addcontentsline{toc}{chapter}{Bibliography}

\begin{singlespace}
  % increase penalty such that we don't break entries over pages
  % source: https://tex.stackexchange.com/a/43275
  \patchcmd{\bibsetup}{\interlinepenalty=5000}{\interlinepenalty=10000}{}{}

  % reduce spacing between each bibentry
  \setlength\bibitemsep{0.9\baselineskip}

  % don't justify-align entries: this prevents stretching out each line
  \raggedright
  \printbibliography[
    heading = none
  ]
\end{singlespace}

\end{document}

%
% note that you'll have to modify the input file to make sure that the
% preamble (\documentclass, etc.) isn't included. to make your life
% easier, you could use some TeX conditionals to make it seamless.
%
% this requires some planning, but enables you to edit the individual
% paper and thesis chapter without tracking and porting changes between
% multiple directories and repositories:
%
% for example, at the beginning of foobar/paper.tex (before
% \documentclass):
%
%   \newif\ifdissertation
%   \dissertationtrue      % (or \dissertationfalse for the standalone)
%
%   \ifdissertation
%   \else
%   \documentclass...
%   \fi
%
%   \ifdissertation
%   \else
%   \begin{document}
%   \fi
%
%   [...paper content here...]
%
%   \ifdissertation
%   \else
%   \end{document}
%   \fi

%%% chapters: lorem ipsum %%%%%%%%%%%%%%%%%%%%%%%%%%%%%%%%%%%%%%%%%%%%%%

% The following text is to test ToC alignment:
% - of extremely long chapter, section, subsection, and
%   subsubsection titles
% - when chapter numbers are double digits

% \chapter{This is a very long title which will take up more than one line
% lorem ipsum dolor sit amet, consectetur adipiscing elit, sed do eiusmod
% tempor incididunt ut labore et dolore magna aliqua. Ultrices vitae
% auctor eu augue ut lectus arcu. Enim nunc faucibus a pellentesque sit
% amet porttitor eget. Consequat mauris nunc congue nisi vitae.}
%   \label{ch:long-title}

% \section{Ut enim ad minim veniam, quis nostrud exercitation ullamco
% laboris nisi ut aliquip ex ea commodo consequat}

% \subsection{Duis aute irure dolor in reprehenderit in voluptate velit
% esse cillum dolore eu fugiat nulla pariatur. Excepteur sint occaecat
% cupidatat non proident, sunt in culpa qui officia deserunt mollit anim
% id est laborum.}

% \subsubsection{Etiam erat velit scelerisque in dictum non. Sit amet
% justo donec enim diam. Amet justo donec enim diam. Metus vulputate eu
% scelerisque felis imperdiet proin. In nulla posuere sollicitudin aliquam
% ultrices. Turpis in eu mi bibendum.}

% \chapter{Lorem Ipsum}
% \chapter{Lorem Ipsum}
% \chapter{Lorem Ipsum}
% \chapter{Lorem Ipsum}
% \chapter{Lorem Ipsum}
% \chapter{Lorem Ipsum}
% \chapter{Lorem Ipsum}
% \chapter{Lorem Ipsum}
% \chapter{Etiam erat velit scelerisque in dictum non. Sit amet
% justo donec enim diam. Amet justo donec enim diam. Metus vulputate eu
% scelerisque felis imperdiet proin. In nulla posuere sollicitudin aliquam
% ultrices. Turpis in eu mi bibendum.}

% \begin{table}
% \centering
% \begin{tabular}{lS}
% \toprule
% $x$      & \textbf{value} \\
% \midrule
% a        & 1.23           \\
% b        & 3.456          \\
% c        & 100.0002       \\
% d        & 12345.0        \\
% \bottomrule
% \end{tabular}
% \caption[In hendrerit gravida rutrum quisque non tellus orci ac. Iaculis
%         urna id volutpat lacus laoreet non curabitur gravida arcu. Mauris
%         ultrices eros in cursus turpis massa. Sed tempus urna et pharetra
%         pharetra massa massa. Eget sit amet tellus cras adipiscing enim eu
%         turpis egestas. Morbi blandit cursus risus at ultrices.]
%         {In hendrerit gravida rutrum quisque non tellus orci ac. Iaculis
%         urna id volutpat lacus laoreet non curabitur gravida arcu.
%         Mauris ultrices eros in cursus turpis massa. Sed tempus urna et
%         pharetra pharetra massa massa. Eget sit amet tellus cras
%         adipiscing enim eu turpis egestas. Morbi blandit cursus risus at
%         ultrices.}
% \label{tbl:example-2}
% \end{table}

%%% chapter 1 %%%%%%%%%%%%%%%%%%%%%%%%%%%%%%%%%%%%%%%%%%%%%%%%%%%%%%%%%%%%%

\chapter{Quid Pro Code: Peer Effects and Productivity in Open Source Software}\label{ch:pg}

\section{Introduction}\label{sec:pg-intro}
\subfile{../pub-goods/intro/intro.tex}

\section{Background}\label{sec:pg-background}
\subfile{../pub-goods/background/background.tex}

\section{Literature}\label{sec:pg-lit}
\subfile{../pub-goods/lit/lit.tex}

\section{Data}\label{sec:pg-data}
\subfile{../pub-goods/data/data.tex}

\section{Reduced Form}\label{sec:pg-reduced}
\subfile{../pub-goods/reduced/reduced.tex}

\section{Structural Model}\label{sec:pg-structural}
\subfile{../pub-goods/struct/struct.tex}

\section{Discussion}\label{sec:pg-discussion}
\subfile{../pub-goods/discussion/discussion.tex}

\clearpage
\begin{subappendices}
\section{Tables}\label{app:pg-tables}
\subfile{../pub-goods/appendix/tables/tables.tex}

\clearpage
\section{Figures}\label{app:pg-figures}
\subfile{../pub-goods/appendix/figures/figures.tex}

\clearpage
\section{Data Details}\label{app:pg-data-app}
\subfile{../pub-goods/appendix/data/data.tex}

\clearpage
% \onehalfspacing
\section{Additional Reduced Form Results}\label{app:pg-reduced-details}
\subfile{../pub-goods/appendix/reduced-details/reduced-details.tex}

\clearpage
% \onehalfspacing
\section{Structural Estimation Details}\label{app:pg-estim-details}
\subfile{../pub-goods/appendix/estim/estim.tex}

\end{subappendices}

%%% chapter 2 %%%%%%%%%%%%%%%%%%%%%%%%%%%%%%%%%%%%%%%%%%%%%%%%%%%%%%%%%%%%%

\chapter{No Free Lunch For Programmers: Digital Supply Chains and the Economics of Software Dependency Management}\label{ch:dsc}

\section{Introduction}\label{sec:dsc-intro}
\subfile{../formation/intro/intro}

\section{Literature}\label{sec:dsc-lit}
\subfile{../formation/lit/lit}

\section{Framework}\label{sec:dsc-framework}
\subfile{../formation/framework/framework}

\section{Data}\label{sec:dsc-data}
\subfile{../formation/data/data}

\section{Reduced Form}\label{sec:dsc-reduced}
\subfile{../formation/reduced/reduced}

\section{Structural Approach}\label{sec:dsc-structural}
\subfile{../formation/structural/structural}

\section{Counterfactual Analysis}\label{sec:dsc-counterfactual}
\subfile{../formation/counterfactual/counterfactual}

\section{Discussion}\label{sec:dsc-discussion}
\subfile{../formation/discussion/discussion}

\clearpage
\begin{subappendices}
\section{Figures}\label{app:dsc-figures}
\subfile{../formation/appendix/figures/figures.tex}

\clearpage
\section{Tables}\label{app:dsc-tables}
\subfile{../formation/appendix/tables/tables.tex}

\clearpage
\section{Mathematical Details}\label{app:dsc-math}
\subfile{../formation/appendix/math/math.tex}

\clearpage
\section{Estimation Details}\label{app:dsc-estimation}
\subfile{../formation/appendix/estimation/estimation.tex}

\end{subappendices}

% %%% conclusions %%%%%%%%%%%%%%%%%%%%%%%%%%%%%%%%%%%%%%%%%%%%%%%%%%%%%%%%
% \chapter{Conclusions}
%   \label{ch:conclusions}

% \graphicspath{}
% \input{conclusions}

%%% bibliography %%%%%%%%%%%%%%%%%%%%%%%%%%%%%%%%%%%%%%%%%%%%%%%%%%%%%%%
%
%  \printbibliography in biblatex is great, but doesn't allow for the
%  greatest customization, so we'll use the package biblatex + biber
%  backend to meet some requirements:
%
%  * bibliography should be an un-numbered chapter, and still have a
%    pdfbookmark and a line in the table of contents
%
%  * bibliography contents should be singlespace, and optionally a smaller
%    font
%
%  * first line of this "chapter" should be in the same spot as the first
%    line of preface sections (e.g., acknowledgement)
%
%  * we use \raggedright so things like URLs and DOIs aren't stretched out.
%
\clearpage
\chapter*{Bibliography}
\addcontentsline{toc}{chapter}{Bibliography}

\begin{singlespace}
  % increase penalty such that we don't break entries over pages
  % source: https://tex.stackexchange.com/a/43275
  \patchcmd{\bibsetup}{\interlinepenalty=5000}{\interlinepenalty=10000}{}{}

  % reduce spacing between each bibentry
  \setlength\bibitemsep{0.9\baselineskip}

  % don't justify-align entries: this prevents stretching out each line
  \raggedright
  \printbibliography[
    heading = none
  ]
\end{singlespace}

\end{document}
