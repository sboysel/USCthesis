%%\lipsum[2-3]

In this manuscript, we explore microeconomic behavior shaping the production of
open source software (OSS).  We fortify our analysis with economic structure to
guide our narrative and assess our hypotheses empirically, filling important
gaps in the literature on the supply side of markets for OSS goods.  Our
motivation is rooted in a desire to better understand how various microeconomic
phenomena influence sustained development of widely used OSS infrastructure.
Following an introduction in Chapter~\ref{ch:introduction}, our contribution is
divided into two distinct chapters.

In Chapter~\ref{ch:pg}, we examine the extent to which peer effects influence
the private provision of public goods.  In the case of public information goods,
peer contribution may facilitate or otherwise incentivize further contribution
from others, effectively subsidizing private provision.  We first utilize a
reduced form approach to derive causal estimates of net peer effects in public
goods contribution by exploiting a peers-of-peers identification strategy.  We
next develop a structural model of peer-influenced public good provision that
both (1) separates extensive and intensive margin contribution decisions and (2)
decomposes contribution into marginal private benefits and costs.  We apply
these methodologies using a sample of peer contribution histories for 2,287 OSS
projects hosted on the GitHub collaboration platform.  Both reduced form and
structural approaches suggest peer effects are much stronger along the extensive
margin than the intensive margin.  Contemporaneous intensive margin effects,
while heterogenous across time and projects, are small and centered around zero,
suggesting that strategic complementarity and substitution in peer contribution
likely offset one another.  Our counterfactual analysis suggests (extensive
margin) peer effects account for nearly 56\% of cumulative aggregate
contribution for our sample, which translates to a value-added of 1-1.5 million
software developer labor hours.  These results support the notion that OSS is
largely developed by disproportionate efforts from smaller groups of dedicated
core maintainers, who integrate incremental contributions from the wider
community, and casts doubt on the potential for peer effects alone to deliver
sustained maintenance labor to individual projects.

In Chapter~\ref{ch:dsc}, we turn our attention to the formation of software
dependency networks.  Developers of software projects can leverage the
functionality of existing open source projects.  This practice can potentially
lower the cost of development albeit at the inherent risk of relying on external
components. A ``downstream'' project maintainer can choose to ``import''
elements of an ``upstream'' project to outsource functionality, but is uncertain
how future changes in this dependency project may expose her own project to
software faults or vulnerabilities. Software dependency networks therefore
represent a ``digital supply chain'', an ecosystem of interdependent public
goods that confer an intricate set of both positive and negative externalities
for project maintainers and end users.  Focusing on microeconomic fundamentals
of the dependency management problem faced by the risk averse project
maintainer, we use both reduced form and structural approaches to study how
dependency networks create value, what forces shape their formation, and how
individual behavior can influence the robustness of equilibrium network
structure.  We use a sample of open source software projects from the Node.js
JavaScript packaging ecosystem for which contribution and dependency formation
decisions are observed in real-time. Finally, we consider several policy
interventions that can improve equilibrium welfare.  In particular, we find that
removing less that 1\% of core projects can reduce aggregate project quality by
more than 5\% for the remaining peers.

